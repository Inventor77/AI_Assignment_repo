\documentclass[13pt,letterpaper]{article}
\usepackage[utf8]{inputenc}

\usepackage{graphicx}
\setlength{\parskip}{0.4em}
\setlength{\parindent}{0em}

\begin{document}
\title{
\includegraphics[width=0.3\textwidth]{National_Institute_of_Technology,_Raipur_Logo.png}
\\ 
Gödel's incompleteness theorems
}

\author{SHREEDUTT 19111056 BME 5TH SEMESTER}

\maketitle
\rule{\textwidth}{0.5pt}
\nolinebreak[4]
\section*{Summary of Theorem}
Mathematical logic as a subject really came of age with the Gödel incompleteness theorem, which shows that in any sufficiently strong formal system in mathematics, there will be true statements that are not provable, and the consistency of such theories is one of those true statements that are not provable.
\\
The Incompleteness theorem demonstrates, first and foremost, that we can never enumerate a complete theory. We cannot list a theory that will prove all and only the true statements of mathematics. Furthermore, we can never prove the consistency of a finitary theory, let alone the consistency of a much stronger infinitary theory. The first and second Incompleteness theorems would be as described in the following.
\\
The incompleteness theorem is a decisive rejection of the Hilbert scheme.The theorems are interpreted as illustrating the incapability of Hilbert's initiative to find a complete and consistent set of axioms for all mathematics.
\\
\subsection* {First Incompleteness Theorem} "Any consistent formal system F within which a certain amount of elementary arithmetic can be carried out is incomplete; i.e., there are statements of the language of F which can neither be proved nor disproved in F."
There is no computably enumerable set of axioms proving all and only the true statements of elementary mathematics.
\\
\subsubsection* {Another version of Incompleteness}
For any computably axiomatized true theory of arithmetic there is  a polynomial with integer coefficient in several variables such that the question  of  whether it has solution in the integers is not provable in the theory.
This theorem is known as MRDP Theorem, its named for Maria Saviac Robinson Davis and Putnam who proved it. 
\subsection* {Second Incompleteness Theorem} "Assume F is a consistent formalized system which contains elementary arithmetic. Then F for all Cons(F)."
It says that if you have a consistent computably theory extending its own consistency. consistency of a theory is a syntactic property of the theory.
This theorem is more powerful than the first incompleteness theorem because the statement constructed in the first incompleteness theorem does not express the system's consistency directly. By formalising the proof of the first incompleteness theorem within the system F on its own, the proof of the second incompleteness theorem is obtained.
\subsubsection* {What did we believe before Gödel?} 
There was a feeling that we should be able to prove that mathematics is consistent. This idea was inspired by a simple paradox that people like Bertand Russell had devised. People may even have learned of "the rest of all sets that do not contain themselves as members."
\subsubsection* {What did Gödel do?}
Gödel devised a method for allowing Mathematics to speak for itself, resulting in the Gödel coding system. Every statement about Mathematics has a unique code number. So, in mathematics, every statement can be converted into a number, and every code number can be worked backwards into a statement. Just like 
computer machine's compiler converts the code into binary, and from binary, the machine can read or display it back.
There will not be a meaningfu l Mathematical statement for every number in the Gödel Code.
\end{document}
