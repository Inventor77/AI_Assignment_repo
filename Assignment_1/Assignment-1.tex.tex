\documentclass[13pt,letterpaper]{article}
\usepackage[utf8]{inputenc}

\usepackage{graphicx}
\setlength{\parskip}{0.4em}
\setlength{\parindent}{0em}

\begin{document}
\title{
\includegraphics[width=0.3\textwidth]{National_Institute_of_Technology,_Raipur_Logo.png}
 \\ 
Philosophy of Artificial Intelligence
}

\author{SHREEDUTT 19111056 BME 5TH SEMESTER}

\maketitle
\rule{\textwidth}{0.5pt}
\nolinebreak[4]
\section*{Summary of Wikipedia Article}
Artificial intelligence is a subfield of technology philosophy that investigates artificial intelligence and its implications for knowledge and understanding of intellectual ability, morality, awareness, theory of knowledge, and unlimited will.
The philosophy of artificial intelligence attempts to answer questions such as:
\nolinebreak[4]
\begin{itemize}
  \item Is it possible for a machine to act intelligently?
  \item Is it possible to feel how things are?
  \item Is it possible for a machine to have a mind, psychological processes, and conscious experience in the same way that a human being does?
\end{itemize}
And many more such questions reflect the diverging interests respectively of AI researchers, cognitive researchers and philosopher. The scientific answers to these questions are dependent on how "intelligence" and "consciousness" are defined, as well as which "machines" are under consideration.\\
\subsubsection* {Can a machine display general intelligence?}
Is it possible to build a machine that can solve all of the problems that humans do with their intelligence? This question defines the scope and direction of the Advanced research in the future. It concerns only the behaviour and the issues of psychology,  scientists and philosophers' interest; it matters to answer this question whether a machine actually thinks (as a person thinks) or acts just like it thinks.
\nolinebreak[4]
\subsubsection* {Can a machine have a mind, consciousness, and mental states?}This is a question of philosophy, related to other minds' problems and to the 
\nolinebreak[4]difficult problem of consciousness. The question focuses on a position defined as "Strong AI" by John Searle: a physical symbolic system can have a state of mind and a mental state.
Searle differentiated this position from what was known as "weak AI": an intelligent physical system of symbols can act.
Searle introduced the terms to isolate strong AI from weak AI to focus on what he considered to be the most interesting and controversial problem. He argued that although we assume that we have a computer programme which acts exactly like a human mind, a difficult philosophical question remains to be answered.
\nolinebreak[4]
\subsubsection* {Is thinking a kind of computation?}
Mental or computational theory states that the relationship between the mind and the brain is similar (if not identical) to the relationship between a running programme and the computer. Hilary Putnam and Jerry Fodor are associated with the latest version.
This question relates to our previous questions: if the human mind is a computer of a kind, then computers are able to answer practical as well as philosophical questions of AI both intelligently and sensitively.
\nolinebreak[4]
\subsubsection* {Other related questions can be}
Is it possible to build a machine that can solve all of the problems that humans do with their intelligence? This question defines the scope and direction of the Advanced research in the future. It concerns only the behaviour and the issues of psychology,  scientists and philosophers' interest; it matters to answer this question whether a machine actually thinks (as a person thinks) or acts just like it thinks.
\nolinebreak[4]
\begin{itemize}
  \item Is it possible for a machine to feel emotions?
  \item Is it possible to feel how things are?
  \item Is it possible for a machine to be self-aware?
  \item Is it possible for a machine to mimic all human characteristics?
  \item Is it possible for a machine to be creative and original?
\end{itemize}

\end{document}
