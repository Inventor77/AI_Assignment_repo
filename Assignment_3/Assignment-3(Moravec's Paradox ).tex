\documentclass[13pt,letterpaper]{article}
\usepackage[utf8]{inputenc}

\usepackage{graphicx}
\setlength{\parskip}{0.4em}
\setlength{\parindent}{0em}

\begin{document}
\title{
\includegraphics[width=0.3\textwidth]{National_Institute_of_Technology,_Raipur_Logo.png}
\\ 
Moravec's Paradox 
}

\author{SHREEDUTT 19111056 BME 5TH SEMESTER}

\maketitle
\rule{\textwidth}{0.5pt}
\nolinebreak[4]
\section*{Summary of Paradox}
In the 1980s, \textbf{Moravec}, along with other pioneering scientists, such as Marvin Minsky and Rodney Brooks, made an observation: “\textbf{\textit{contrary to traditional assumptions, reasoning (which is high-level in humans) requires very little computation, but sensorimotor skills (comparatively low-level in humans) require enormous computational resources}}”.
\\
As Moravec writes, “it is comparatively easy to make computers exhibit adult level performance on intelligence tests or playing checkers, and difficult or impossible to give them the skills of a one-year-old when it comes to perception and mobility”.
\\
Marvin Minsky agreed with him, stating that unconscious human skills are the most difficult to reverse engineer. "\textbf{\textit{In general, we're least aware of what our minds do best," he wrote, and "we're more aware of simple processes that don't work well than complex ones that work flawlessly.}}"
\\
\subsection* {The biological basis of human skills} 
All human skills are implemented biologically, using machinery designed by the process of natural selection. It is also a process that places a lot of value on optimization (of energy expenditure).
\\
As a result, human skill development throughout evolution has tended to preserve design improvements that meet extremely stringent optimization criteria. This optimization process, however, is slow and takes a long time to shape and guide the evolution of a specific skill.
As a matter of fact, it is simply understood that the older a skill is, the more time natural selection has had to improve the design.
\\
Because abstract thought and high-level mathematical reasoning are relatively new, their implementation and mechanics are not expected to be particularly efficient.
\\
However, this is a relatively new development.
Motor skills and sensory perception had aided the survival and nourishment of our genes for hundreds of millennia before that. Only those with superior reflexes and the ability to spot a predatory threat from afar, or who could run faster and throw with pinpoint accuracy, survived and passed on their genes.
\\
This led them to believe that once these ‘hard’ problems are solved completely by AI, the ‘easy’ problems of ‘recognizing a friend’s face from a photo’ or ‘moving around a college dorm room without tipping over’, should be a walk in the park.
\subsection* {Early years of AI research}
Rodney Brooks puts it, the intelligent things were “best characterized as the things that highly educated male scientists found challenging” — chess, symbolic integration, proving mathematical theorems, and solving complicated textual algebra problems. “The things that children of four or five years could do effortlessly, such as visually distinguishing between a coffee cup and a chair, or walking around on two legs, or finding their way from their bedroom to the living room were not thought of as activities requiring intelligence”.

\subsection* {Conclusion}
Whatever the case, we saw that there is a strong scientific argument to believe that a wide range of jobs requiring complex physical movement and subtle sensory perception (e.g., gardening, barbering, elder care, childcare, sports, bartending, hotels and hospitality) are unlikely to be replaced by robotic automation anytime soon. These jobs, while appearing to be simple on the surface, rely on skills that have been refined over millions of years of biological process optimization, and require far more computations than many white-collar repetitive jobs that rely solely on high-level reasoning, a relatively recent product of human evolution.
\end{document}
