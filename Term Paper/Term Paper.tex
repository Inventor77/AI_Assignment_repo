  \documentclass[12pt,letterpaper]{article}

  \usepackage[utf8]{inputenc}
  \usepackage{graphicx}
  \setlength{\parskip}{.8em}
  \setlength{\parindent}{0em}

  \begin{document}
  \topskip0pt
\begin{center}

  \includegraphics[width=.5\textwidth]{National_Institute_of_Technology,_Raipur_Logo.png}

    \hrule
    \Huge
    \textbf{SHREEDUTT DIXIT}
    \\
    \Large
    (19111056)
    \\
    ..........................................................................
    \\
    \textbf{BIOMEDICAL ENGINEERING}
    \\
    5th SEMESTER
    \\
    ..........................................................................
    \\
    \Large
    \textbf{TERM PAPER TOPIC :}

    \large
    BLOCKCHAIN FOR AI \\
    \_
    \hrule
\end{center}
\pagebreak
\large
\begin{center}
    \section* {Introduction}
\end{center}
\textbf{AI} and \textbf{blockchain} are two of the most important drivers of innovation today. Both are bringing about significant changes in all aspects of our lives and are expected to contribute trillions of dollars to the global economy. 
  \\
  \\
  The future has arrived, complete with self-driving cars and charming assistants who can schedule appointments on your behalf in natural conversations. Furthermore, the emergence of new content and economic sharing platforms will mean that users will no longer be forced to rely on “untrustworthy middlemen” such as Facebook. 
  \\
  \\
  So, what happens if we combine these two disruptive technologies? Following the introduction of each new tech, we will share our vision for their combined future.

  \pagebreak
  Blockchain can power decentralized marketplaces and coordination platforms for various components of AI, including data, algorithms, and computing power. These will foster the innovation and adoption of AI to an unprecedented level.
  \\ 
  \\
  Blockchain will also help AI’s decisions be more transparent, explainable, and trustworthy. As all data on blockchain is publicly available, AI is the key to providing users with confidentiality and privacy.

\pagebreak

  \hrule

  \section* {Table Of Content}
    \begin{itemize}
      \item Introduction
      \item Abstract
      \item Secure Data Sharing
      \item Selling spare Computing power
      \item AI in Healthcare 
      \item Blockchain in Healthcare
      \item Conclusion
    \end{itemize}

    \hrule

\pagebreak

    \subsection* {Secure Data Sharing}
    The vast amount of data available for research, development, and commerce is one of the driving forces behind the current AI revolution. Nowadays, in today's data-driven economy, data is the new gold.
    \\
    \\
    However, getting your hands on this gold has a lot of obstacles. For starters, unless you're working for a major player like Google, it might be difficult to collect enough data for your training models. Ultimately, this reduces the level of competition among AI researchers and firms, which is necessary to advance AI technology. 
    \\
    \\
    Second, privacy has become a major issue as a result of numerous data breaches and misuses. the recent Facebook scandal, in which 50 million people were profiled and targeted without their knowledge or consent.
    \\
    \\
    It's possible that blockchain will encourage data sharing because it makes it clear who has access to what data and when. Users will have more confidence in sharing data as blockchain returns control of data to the users. 
    \\
    \\
    They will also know that their data will be used appropriately to improve professionalization or other good causes. Medical records and cases could be made available to doctors and researchers, speeding up the search for cures and the development of better medical procedures.
    \\
    \\
    Doctors would have access to similar cases from around the world, giving patients with rare diseases new hope.

\pagebreak


    \subsection* {Selling Spare Computing Power}
    Blockchain can leverage more distributed computing power for AI by creating a decentralised market for computing power, i.e. blockchain-based cloud computing. AI developers can now use millions of GPUs from gamers to prepare, model, train, and deploy machine learning algorithms.
    \\
    \\
    Gamers, whose GPUs are frequently used for a short period of time, can list their computing time for bids in the form of AI smart contracts and be paid.
    \\
    \\
    You are correct if you believe that the concept of selling surplus computing power is not novel. Grid computing, a once-popular idea, found limited application and was not intended for mass adoption. In contrast, today's market for AI-supported cloud computing has as many applications as AI can power. 
    \\
    \\
    Encouraging users to sell their computing resources via cryptography token payment, on the other hand, encourages more users to sell their computing resources.

\pagebreak

\subsection* {AI in Healthcare}
    AI is becoming more sophisticated at doing what humans do, but more efficiently, quickly, and cheaply. Both AI and robotics have enormous potential in healthcare. AI and robotics are increasingly becoming a part of our healthcare eco-system, just as they are in our daily lives.
\\
\\
One of the most significant potential benefits of AI is that it can help people stay healthy so that they don't need to see a doctor as often, if at all. People are already benefiting from the use of AI and the Internet of Medical Things (IoMT) in consumer health applications.
\\
\\
Individuals are encouraged to adopt healthier behaviours by using technology applications and apps, which aid in the proactive management of a healthy lifestyle. It gives consumers control over their health and well-being.
\\
\\
Furthermore, AI improves healthcare professionals' ability to better understand the day-to-day patterns and needs of the people they care for, allowing them to provide better feedback, guidance, and support for staying healthy.

    % \subsection*{Blockchain in Healthcare}

    
  \end{document}
